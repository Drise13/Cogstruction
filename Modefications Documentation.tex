\documentclass[12pt, letterpaper]{article}

% Language setting
% Replace `english' with e.g. `spanish' to change the document language
\usepackage[english]{babel}

% Set page size and margins
% Replace `letterpaper' with `a4paper' for UK/EU standard size
\usepackage[a4paper,top=2cm,bottom=2cm,left=1cm,right=1cm,marginparwidth=1.75cm]{geometry}

% Useful packages
\usepackage{amsmath}
\usepackage{graphicx}
\usepackage[colorlinks=true, allcolors=blue]{hyperref}

\title{Cogstruction modefications documentation}
\author{Lily Stejko}

\begin{document}
\maketitle

\begin{abstract}
Don't need one, just for taking notes.
\end{abstract}

\section{Learning algorithm weights}

\subsection{Original fitness function}

Shorthand reference for section

$SOF =$ standard\_obj\_fxn

\textbf{Inputs}  

$iB_w =$ inv\_build\_weight $= 7000.0$\footnote[1]{The reason for giving the variables their starting weights is not known.}

$iF_w =$ inv\_flaggy\_weight $= 2000.0$\footnotemark[1]

$iE_w =$ inv\_exp\_weight $= 3.0$\footnotemark[1]

\textbf{Weights}

$B_w =$ build\_weight

$F_w =$ flaggy\_weight

$E_w =$ exp\_weight

\textbf{Object parameters}

$B_r =$ build\_rate

$F_r =$ flaggy\_rate

$E_r =$ exp\_mult

\subsubsection{Fitness function}

$SOF(B_w,F_w,E_w) = B_r*B_w+F_r*F_w+E_r*E_w$

\subsubsection{Calculating weights}

Using the following variables.

\textbf{Formula}

$\begin{bmatrix}
1 & 1 & 1\\
iB_w & -iF_w & 0.0\\
0.0 & iF_w & -iE_w
\end{bmatrix}
\begin{bmatrix}
B_w\\
F_w\\
E_w
\end{bmatrix}
=
\begin{bmatrix}
1\\
0\\
0
\end{bmatrix}$

\textbf{Linear algebra form}

$B_w+F_w+E_w = 1$ (Makes sure all values sum up to 1)

$iB_w*B_w-iF_w*F_w = 0$ (Scale lock the ratio between $B_w$ and $F_w$)

$iF_w*F_w-iE_w*E_w = 0$ (Scale Lock the ratio between $F_w$ and $E_w$)

\clearpage

\subsection{New fitness function}

Shorthand reference for section

$F =$ Fitness Formula\

\textbf{Inputs}

The input weights are input at the ratio desired.

$B_{w_i} =$ input\_build\_weight

$F_{w_i} =$ input\_flaggy\_weight

$E_{w_i} =$ input\_exp\_weight

$Adj_v =$ adjust\_parameters\footnote{A bool value used to determine if the values of B\_r, F\_r, and E\_r should be adjusted to reflect their non-linear relationships to gear crafter levels.} \textit{(True by default)}

\textbf{Weights}

$B_w =$ build\_weight

$F_w =$ flaggy\_weight

$E_w =$ exp\_weight

\textbf{Object parameters}

$B_{r_c} =$ build\_rate for a given cog

$F_{r_c} =$ flaggy\_rate for a given cog

$E_{r_c} =$ exp\_mult for a given cog

$B_r =$ build\_rate for a cog array

$F_r =$ flaggy\_rate

$E_r =$ exp\_mult

$SQ_b =$ usable\_suares

\textbf{Constants and calculated variables}

$C =$ Construction Value

$Af =$ Affixes \textit{(5.5 for current best cogs, $5+50\%$ chance of a sixth)}

$Af_b =$ Base Affixes \textit{(5 for current best cogs)}

$Sn =$ Number of stats on a gear that an affix can be assigned to \textit{(Known const, 3)}

$b =$ Expected booster source ratio \textit{(assumed 2)\footnote{for simplicity, indicating that approximately $\frac{1}{2}$ of the expected value of the grid comes from cog base values}}

\subsubsection{Fitness function}

$F = B_w*\frac{Sn*B_r}{Af*SQ_b}+F_w*b\left(\frac{Sn*F_r}{Af*SQ_b*b}\right)^{1.25}+$

$E_w*b\left(\left(0.0003808514*\left(\frac{Sn*E_r}{Af*SQ_b*b}\right)^{3.968829}\right)+\left(\frac{Sn*E_r}{Af*SQ_b*b}\right)^{1.075056}+0.8517081\right)$

Applying the known values:


$F = B_w*\frac{3*B_r}{5.5*SQ_b}+F_w*2*\left(\frac{3*F_r}{11*SQ_b}\right)^{1.25}+$

$E_w*2*\left(\left(0.0003808514*\left(\frac{3*E_r}{11*SQ_b}\right)^{3.968829}\right)+\left(\frac{3*E_r}{11*SQ_b}\right)^{1.075056}+0.8517081\right)$

\subsubsection{Calculating weights}

$[B_w,F_w,E_w] = \frac{[B_{w_i}, F_{w_i}, E_{w_i}]}{B_{w_i} + F_{w_i} + E_{w_i}}$

\subsubsection{Fitness function derivation}

The "construction value" ($C$) is the base value that determines the power of an single Affix for a given gear type crafted at a given construction level. The formula to calculate the construction value at a given skill level is:

$C=\frac{\left(\left(\frac{level}{3}\right)+0.7\right)^{1.3+0.05*Af_b}}{4} + 3^{Af_b-2}$

The actual bonus for a given affix can range from 40\% to 300\% of that value.

\textbf{Formulas for $B_r$, $F_r$, and $E_r$:}

$B_{r_c} = C$, the relationship is 1 to 1 (and therefor linear)

$F_{r_c} = C^{0.8}$, the relationship is not linear.

$E_{r_c} = C^{0.4}+LOG10(C)*10-5$, the relationship is not linear.

\textbf{The inverse formulas are:}

$C = B_{r_c}$

$C = F_{r_c}^{1.25}$

$C = b*388.428W\left(0.145974*\sqrt[25]{10^{E_{r_c}}}\right)^{\frac{5}{2}}$

The last formula is too complex to use in practice but

$C = (0.0003808514*E_{r_c}^{3.968829})+E_{r_c}^{1.075056}+0.8517081$

is much faster to calculate and is a good approximation formula.

\textbf{The cog array:}

Every cog array has the same number of cogs, $SQ_b$.

To get the average values of an average cog we take  $\frac{[B_r,F_r,E_r]}{SQ_b}=\left[\frac{B_r}{SQ_b},\frac{F_r}{SQ_b},\frac{E_r}{SQ_b}\right]$

We can generally assume top level cogs for every cog, if the player is using lower level cogs, that's ok! This formula is just trying to get some way to bring $B_r$, $F_r$, and $E_r$ into an approximately consistent range.

To calculate the average affix, multiply by $\frac{Sn}{Af}$: $\left[\frac{Sn*B_r}{Af*SQ_b},\frac{Sn*F_r}{Af*SQ_b},\frac{Sn*E_r}{Af*SQ_b}\right]$

Since we assume that $\frac{1}{b}$ of the effect comes from a bonus:

$\left[\frac{Sn*B_r}{Af*SQ_b*b},\frac{Sn*F_r}{Af*SQ_b*b},\frac{Sn*E_r}{Af*SQ_b*b}\right]$

Applying the inverse formulas and reapplying b to the final result:

$\left[\frac{Sn*B_r}{Af*SQ_b},b\left(\frac{Sn*F_r}{Af*SQ_b*b}\right)^{1.25},b*388.428W\left(0.145974*\sqrt[25]{10^{\frac{Sn*E_r}{Af*SQ_b*b}}}\right)^{\frac{5}{2}}\right]$

Using the more efficient approximation of the last formula 

$\begin{left}
\left[\frac{Sn*B_r}{Af*SQ_b},b\left(\frac{Sn*F_r}{Af*SQ_b*b}\right)^{1.25},b\left(\left(0.0003808514*\left(\frac{Sn*E_r}{Af*SQ_b*b}\right)^{3.968829}\right)+\left(\frac{Sn*E_r}{Af*SQ_b*b}\right)^{1.075056}+0.8517081\right)\right]
\end{left}$

These can now be multiplied by $\left[B_w,F_w,E_w\right]$ and the result can be summed.

\end{document}